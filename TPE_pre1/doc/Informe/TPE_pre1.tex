 
\documentclass[%
	%draft,
	%submission,
	%compressed,
	final,
	%
	%technote,
	%internal,
	%submitted,
	%inpress,
	reprint,
	%
	%titlepage,
	notitlepage,
	%anonymous,
	narroweqnarray,
	inline,
	twoside,
	invited
	]{ieee}

\usepackage[utf8]{inputenc}
\usepackage[spanish]{babel}
\usepackage{graphicx}
\usepackage{verbatim}
\usepackage{moreverb}
\usepackage{amsmath}
\usepackage{amsfonts}
\usepackage{amssymb}
\usepackage{fancybox}
\usepackage{float}
\usepackage{fancyvrb}
\usepackage{subfigure}

\newcommand{\latexiie}{\LaTeX2{\Large$_\varepsilon$}}

%\usepackage{ieeetsp}	% if you want the "trans. sig. pro." style
%\usepackage{ieeetc}	% if you want the "trans. comp." style
%\usepackage{ieeeimtc}	% if you want the IMTC conference style

% Use the `endfloat' package to move figures and tables to the end
% of the paper. Useful for `submission' mode.
%\usepackage {endfloat}

% Use the `times' package to use Helvetica and Times-Roman fonts
% instead of the standard Computer Modern fonts. Useful for the 
% IEEE Computer Society transactions.
%\usepackage{times}
% (Note: If you have the commercial package `mathtime,' (from 
% y&y (http://www.yandy.com), it is much better, but the `times' 
% package works too). So, if you have it...
%\usepackage {mathtime}

% for any plug-in code... insert it here. For example, the CDC style...
%\usepackage{ieeecdc}

\begin{document}

%----------------------------------------------------------------------
% Title Information, Abstract and Keywords
%----------------------------------------------------------------------
\title[Métodos de búsqueda no informados]{%
       Métodos de búsqueda no informados}

% format author this way for journal articles.
% MAKE SURE THERE ARE NO SPACES BEFORE A \member OR \authorinfo
% COMMAND (this also means `don't break the line before these
% commands).
\author[Castiglione, Karpovsky, Sturla]{Gonzalo V. Castiglione, Alan E. Karpovsky, Martín Sturla\\\textit{Estudiantes 
       Instituto Tecnológico de Buenos Aires (ITBA)}\\
\\\textbf{15 de Marzo de 2012}
}



\journal{Cátedra\ \ Sist.\ de\ Inteligencia\ Artificial,\ ITBA\ }
\titletext{-\ 15, MARZO\ 2012}
\ieeecopyright{\copyright\ 2011 ITBA}
\lognumber{}
\pubitemident{}
\loginfo{15 de Marzo, 2012.}
\firstpage{1}

\confplacedate{Buenos Aires, Argentina, 15 de Marzo, 2012}

\maketitle               

\begin{abstract} 
El presente informe busca analizar y comparar distintas estrategias de búsqueda no informadas sobre un problema en particular (Skyscraper puzzle) haciendo uso de un motor de inferencias.
\end{abstract}

\begin{keywords}
DFS, BFS, General Problem Solver, depth-first, breadth-first, search strategy
\end{keywords}

%----------------------------------------------------------------------
% SECTION I: Introduccion%----------------------------------------------------------------------
\section{Introducción}

\PARstart El juego \textit{Edificios} también conocido como \textit{Skyscraper puzzle} es una variante del conocido \textit{Sudoku} y consiste de una grilla cuadrada con números en su borde que representan las pistas sobre cuántos edificios se visualizan en esa dirección. El tablero, visto desde arriba, representa un espacio cubierto de edificios. Cada casillero debe ser completado con un dígito que va entre 1 y N, siendo N el tamaño de la grilla; haciendo que cada fila y cada columna contengan sólo una vez a cada dígito (como sucede en el Sudoku).

\par En este puzzle, cada dígito puesto en la grilla podría ser visualizado como un edificio de esa altura. Por ejemplo, si ingresamos un 5, estaríamos colocando un edificio de altura 5. Cada uno de los numeros que están por fuera de la grilla revelan la cantidad de edificios que pueden ser vistos al mirar la línea o columna en esa dirección. Cada edificio bloquea la visión de todo edificio de menor altura que se encuentre detrás de él.

%----------------------------------------------------------------------
% SECTION II: Marco Teórico
%----------------------------------------------------------------------

\section{Estados del problema}

\subsection{Estado inicial}

\par El estado inicial del problema es un tablero que contiene sólo las pistas en sus bordes (no contiene ningún edificio en la grilla). Para que el problema tenga solución éste tablero debe ser válido; es decir que no cualquier combinación de pistas sobre la visibilidad de edificios en esa dirección conducirán a un problema resoluble.

\subsection{Estado final}

\par El estado final del problema es un tablero con todos los casilleros completos (lleno de edificios) que cumpla con las reglas del juego citadas anteriormente.

\section{Modelado del problema}

\par El tablero del juego se modeló mediante la creación de la clase Board; la misma contiene una variable privada de tipo matriz de enteros que representa a la grilla propiamente dicha. Si un casillero dicha matriz  tiene un $0$, significa que  está vacío; si en cambio tiene un número $k \in [1,n]$ significa que hay situado un edificio con altura $k$.\\
\par Las pistas o restricciones del tablero (números en los bordes del mismo) se modelaron con un arreglo de cuatro vectores de enteros (TOP, BOTTOM, LEFT, RIGHT) que simplemente contienen la restricción en esa dirección.
Cabe destacar que en los estados únicamente se guarda el tablero; guardar las restricciones sería redundante. Ver \textbf{Figura \ref{modelado}} en la sección \textbf{Anexo B}.\\



\section{Reglas}

\par Dado un tablero de tamaño $n\times n$ podemos describir las reglas del problema en forma general como sigue:\\

\emph{Poner un edificio de altura $x$ en la posición $(i,j)$ del tablero, con $x \in [1, n]$.}\\

\par Por lo que, que dado un tablero de $n\times n$, el total de reglas está dado por $n \times {n \times n} = n^3$ ya que por cada fila y por cada columna, se tienen $n$ edificios distintos para colocar. 
\par Ejemplos de reglas:
\begin{itemize}
\item Poner un edificio de altura 1 en la posición $(0,0)$
\item Poner un edificio de altura 1 en la posición $(0,1)$
\item Poner un edificio de altura 2 en la posición $(0,0)$
\item Poner un edificio de altura 4 en la posición $(3,1)$
\item $\ldots$

\end{itemize}

\par Las reglas así definidas producen un factor de ramificación de $n^3$ lo que hace que, para tableros grandes (mayores a $5\times5$) se tarde un tiempo muy considerables (en el orden de las horas) en encontrar la solución. Para subsanar este inconveniente se puede plantear el problema enunciando con reglas diferentes, enumeradas a continuación:\\

\emph{Poner un edificio de tamaño $a$ en el próximo lugar disponible de izquierda a derecha y de arriba hacia abajo.}\\

\par En un tablero de $n\times n$ el conjunto de reglas quedaría definido de la siguiente forma:\\

\begin{itemize}
\item Poner un edificio de altura 1 en el próximo lugar disponible de izquierda a derecha y de arriba hacia abajo.
\item Poner un edificio de altura 2 en el próximo lugar disponible de izquierda a derecha y de arriba hacia abajo.
\item $\vdots$
\item Poner un edificio de altura $n$ en el próximo lugar disponible de izquierda a derecha y de arriba hacia abajo.
\end{itemize}

\par Esto nos otorga un factor de ramificación igual a $n$ siendo este inferior al del caso anterior. Esta estrategia reduce la cantidad de reglas como así también el factor de ramificación, se tomó la decisión de implementar ambos conjuntos de reglas y comparar los resultados.
La tablas que se encuentran en el anexo ilustran pruebas que demuestran que el conjunto de reglas reducido es superior al otro como se sospechó.

\section{Costos}

\par Debido a la naturaleza del probema, la aplciacion de todas las reglas tiene el mismo costo y éste es unitario. Es decir que la transición de un estado X a un estado Y, en nuestro caso, siempre nos cuesta $1$.

\section{Algoritmos de búsqueda implementados}

\par Hemos implementado cuatro algoritmos de búsqueda no informada diferentes. Los mismos son:

\begin{description}

\item[DFS]
Depth-first search

\item[BFS]
Breadth-first search

\item[IDFS]
Iterative DFS o profundización iterativa

\item[HIDFS]
Hybrid Iterative DFS

\end{description}

\section{Heurísticas}

\par La idea detrás de las heurísticas, en un  problema de satisfacción de restricciones como el discutido, no es hallar el camino óptimo a la solución ya que previamente se conoce su profundidad y  costo,  sino descartar caminos que no conducirán al algoritmo hacia la solución. Esto se logra, para un determinado nodo, asignándole el valor "infinito" a la función $h(n)$ de los sucesores irresolubles de éste.\\

\subsection{Heurística 1}

\par Si una restricción indica el número $1$, esa fila o columna deberá comenzar con la altura maxima $(n)$. Además si la restricción opuesta es un $2$, se coloca $n-1$ al lado de este.

\subsection{Heurística 2}

\par Dado un tablero de dimensión $n$, si una restricción indica algún número distinto de $1$ , esa fila o columna trivialmente no podrá comenzar con $(n)$.
\par Si se considera el caso general, si una restricción indica algún número $a$ mayor o igual que $1$ , esa fila o columna deberá tener un edificio con la altura máxima $(n)$ a, al menos, $a$ casilleros de distancia. Esto es fácil de verificar: si 
el edificio de altura máxima está a una distancia de $k$, a lo sumo puede haber $k$ edificios observados. Considerando que ambos lados tienen restricciones (izquierda y derecha ó arriba y abajo), se puede localizar a algunos casilleros donde está la altura máxima. Por ejemplo, si tenemos $K$ como restricción izquierda y $R$ como restricción derecha, esto implica que la altura máxima tiene que estar a, al menos, $R$ casilleros desde la izquierda y a, al menos, $K$ casilleros desde la derecha.
\par Pero aún se puede generalizar más. Sea $k$ la altura de un cierto edificio. Dado una restricción $a$, se verifica que el edificio de altura $k$ debe estar como mínimo a una distancia de $a - n + k$. La demostración es análoga a la anterior. Desde ya, si dicha distancia es menor o igual que uno no hay ningún 
tipo de información útil. Asimismo, cuanto mayor sea este número mayor es la información obtenida: las posibilidades para colocar el edificio con altura $k$ son menores. 
Es fácil ver que $ a + k - (n + 1) $ es la cantidad de casilleros que no pueden contener al edificio de altura $k$, por lo que se aplicará esta heurística únicamente cuando $ a + k - 1 > n$, es decir cuando el número en la restricción y 
la altura del edificio es relativamente alta comparado a $n$. 
\par De esta generalidad se obtiene un resultado útil. Dado una fila o columna con restricción $n$, cada edificio de altura $k$ debe estar al menos $n - n + k = k$ distancia. En otras palabras, el edificio de altura máxima debe estar a $n$ casilleros de distancia, 
el de altura $n-1$ a distancia $n-1$ o mayor, y así sucesivamente. Esto implica que los edificios deben estar ordenados de menor a mayor si se da esta situación.
\section{Resultados}

\par A través de las distintas pruebas realizadas se observó que el algoritmo con mejor desempeño cualquiera sea la dimensión del tablero, es el DFS. Ver \textbf{tabla \ref{tabla1}} en la sección \textbf{Anexo}. A su vez, entendiendo que para este trabajo los métodos debían ser desinformados , fue interesante comparar los tiempos obtenidos en el punto anterior con los tiempos obtenidos dándole un órden particular a las reglas antes de que el \textit{General Problem Solver} comience a resolver. VER ESTO: SE PUEDE O NO!!!??! Básicamente lo que se hizo fue preprocesar las pistas o restricciones del tablero y ordenar el conjujnto de reglas. Una vez ordenadas éstas no son vueltas a tocar y son pasadas al \textit{GPS} para que este las aplique, en ese órden, en cada paso. Ver \textbf{tabla \ref{tabla2}} en la sección \textbf{Anexo}.

\section{Conclusión}

\PARstart Es notable destacar la cantidad tiempo requerido y uso de recursos a para resolver pequeños problemas con metodos no informados. Ya que, luego de ver los resultados obtenidos para pequeños tableros, se necesitó hacer uso de un pre-ordenamiento de las reglas ya que la resolucion del problema tomaba en el orden de las horas para tableros de dimensiones de tan solo 4.

Resultó interesante tambén la forma de resolver problemas mediante la aplicación del mismo conjunto reglas en forma repetida y variando el orden
en cual se agregan los nuevos estados al conjunto de no explorados. 
Comparando los algoritmos no sorprende que el orden de performance haya sido DFS, IDFS y BFS en ese orden. Como la profundidad del problema está acotada 
en la profundidad de la solución, el DFS resulta óptimo dado que no corre el riesgo de exceder la profundidad de la solución óptima o de caer 
en ciclos. Dado que se sabe de antemano que en un tablero 
de dimensión $n \times n$ la profundidad de la solución es $n^{2}$, el IDFS no es más que un DFS con el \textit{overhead} extra de tener que explorar 
todos los nodos con profundidad $n^{2} - 1$. Por último la performance del BFS no es óptima dado que dada la naturaleza exponencial del problema los estados 
que debe explotar y recorrer son simplemente muy abundantes para resolver el problema eficientemente con una busqueda tan abarcativa. 

(continuar!!!)


%----------------------------------------------------------------------
% The bibliography. This bibliography was generated using the following
% two lines:
%\bibliographystyle{IEEEbib}
%\bibliography{ieeecls}
% where, the contents of the ieeecls.bib file was:
%
%@book{lamport,
%        AUTHOR = "Leslie Lamport",
%         TITLE = "A Document Preparation System: {\LaTeX} User's Guide
%                  and Reference Manual",
%       EDITION = "Second",
%     PUBLISHER = "Addison-Wesley",
%       ADDRESS = "Reading, MA",
%          YEAR = 1994,
%          NOTE = "Be sure to get the updated version for \LaTeX2e!"
%}
%
%@book{goossens,
%        AUTHOR = "Michel Goossens and Frank Mittelbach and
%                  Alexander Samarin",
%         TITLE = "The {\LaTeX} Companion",
%     PUBLISHER = "Addison-Wesley",
%       ADDRESS = "Reading, MA",
%          YEAR = 1994,
%}
%
% The ieeecls.bbl file was manually included here to make the distribution
% of this paper easier. You need not do it for your own papers.

%\clearpage

%\begin{thebibliography}{1}

%\bibitem{lamport1}
%Fierens, P. (2011),
%\newblock {\em Cuadrados mínimos: repaso},
%\newblock Buenos Aires: Instituto Tecnológico de Buenos Aires.

%\bibitem{lamport1}
%Abdi, H.,
%\newblock {\em  Least-squares: {Encyclopedia for research methods for the social sciences}},
%\newblock Thousand Oaks (CA): Sage. pp, 2003.

%\bibitem{lamport1}
%Farebrother, R.W. (1988),
%\newblock {\em Linear Least Squares Computations, STATISTICS: Textbooks and Monographs}, %\newblock New York: Marcel Dekker.

%\bibitem{lamport1}
%Lipson, M.; Lipschutz, S. (2001),
%\newblock {\em Schaum's outline of theory and problems of linear algebra}, 
%newblock New York: McGraw-Hill, pp. 69–80.


%\end{thebibliography}

%----------------------------------------------------------------------


\clearpage

\onecolumn

\onecolumn
\section*{Anexo A: Tablas}

\begin{table}[H]
\begin{center}
\begin{tabular}{|c|c|c|c|}
\hline
Algoritmo & Dimensión (n) & Tiempo & Nodos explotados\\
\hline
\hline

DFS & $2\times2$ & - & -\\
DFS & $3\times3$ & - & -\\
DFS & $4\times4$ & - & -\\
DFS & $5\times5$ & - & -\\
\hline
BFS & $2\times2$ & - & -\\
BFS & $3\times3$ & - & -\\
BFS & $4\times4$ & - & -\\
BFS & $5\times5$ & - & -\\
\hline
IDFS & $2\times2$ & - & -\\
IDFS & $3\times3$ & - & -\\
IDFS & $4\times4$ & - & -\\
IDFS & $5\times5$ & - & -\\
\hline
HIDFS & $2\times2$ & - & -\\
HIDFS & $3\times3$ & - & -\\
HIDFS & $4\times4$ & - & -\\
HIDFS & $5\times5$ & - & -\\

\hline  
\end{tabular}
\end{center}
\caption{Comparación de tiempos entre algoritmos para distintos tableros}
\label{tabla1}
\end{table}

\begin{table}[H]
\begin{center}
\begin{tabular}{|c|c|c|c|}
\hline
Algoritmo & Dimensión (n) & Tiempo & Nodos explotados\\
\hline
\hline

DFS & $2\times2$ & - & -\\
DFS & $3\times3$ & - & -\\
DFS & $4\times4$ & - & -\\
DFS & $5\times5$ & - & -\\
\hline
BFS & $2\times2$ & - & -\\
BFS & $3\times3$ & - & -\\
BFS & $4\times4$ & - & -\\
BFS & $5\times5$ & - & -\\
\hline
IDFS & $2\times2$ & - & -\\
IDFS & $3\times3$ & - & -\\
IDFS & $4\times4$ & - & -\\
IDFS & $5\times5$ & - & -\\
\hline
HIDFS & $2\times2$ & - & -\\
HIDFS & $3\times3$ & - & -\\
HIDFS & $4\times4$ & - & -\\
HIDFS & $5\times5$ & - & -\\

\hline  
\end{tabular}
\end{center}
\caption{Comparación de tiempos entre algoritmos con ordenamiento de reglas}
\label{tabla2}
\end{table}

\clearpage

\section*{Anexo B: Imágenes}

\begin{figure}[H]
\begin{center}
\includegraphics[scale=0.65]{./images/ModeladoSIA.jpg}
\label{modelado}
\end{center}
\end{figure}

\begin{center}
\par Figura 1: Modelado del problema
\end{center}


%\VerbatimInput{./code/calculoAb.m}




\end{document}
