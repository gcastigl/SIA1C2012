
\documentclass[%
	%draft,
	%submission,
	%compressed,
	final,
	%
	%technote,
	%internal,
	%submitted,
	%inpress,
	reprint,
	%
	%titlepage,
	notitlepage,
	%anonymous,
	narroweqnarray,
	inline,
	twoside,
	invited
	]{ieee}

\usepackage[utf8]{inputenc}
\usepackage[spanish]{babel}
\usepackage{graphicx}
\usepackage{verbatim}
\usepackage{moreverb}
\usepackage{amsmath}
\usepackage{amsfonts}
\usepackage{amssymb}
\usepackage{fancybox}
\usepackage{float}
\usepackage{fancyvrb}
\usepackage{subfigure}

\newcommand{\latexiie}{\LaTeX2{\Large$_\varepsilon$}}

%\usepackage{ieeetsp}	% if you want the "trans. sig. pro." style
%\usepackage{ieeetc}	% if you want the "trans. comp." style
%\usepackage{ieeeimtc}	% if you want the IMTC conference style

% Use the `endfloat' package to move figures and tables to the end
% of the paper. Useful for `submission' mode.
%\usepackage {endfloat}

% Use the `times' package to use Helvetica and Times-Roman fonts
% instead of the standard Computer Modern fonts. Useful for the 
% IEEE Computer Society transactions.
%\usepackage{times}
% (Note: If you have the commercial package `mathtime,' (from 
% y&y (http://www.yandy.com), it is much better, but the `times' 
% package works too). So, if you have it...
%\usepackage {mathtime}

% for any plug-in code... insert it here. For example, the CDC style...
%\usepackage{ieeecdc}

\begin{document}

%----------------------------------------------------------------------
% Title Information, Abstract and Keywords
%----------------------------------------------------------------------
\title[Métodos de búsqueda no informados]{%
       Métodos de búsqueda no informados}

% format author this way for journal articles.
% MAKE SURE THERE ARE NO SPACES BEFORE A \member OR \authorinfo
% COMMAND (this also means `don't break the line before these
% commands).
\author[Castiglione, Karpovsky, Sturla]{Gonzalo V. Castiglione, Alan E. Karpovsky, Martín Sturla\\\textit{Estudiantes 
       Instituto Tecnológico de Buenos Aires (ITBA)}\\
\\\textbf{15 de Marzo de 2012}
}



\journal{Cátedra\ \ Sist.\ de\ Inteligencia\ Artificial,\ ITBA\ }
\titletext{-\ 15, MARZO\ 2012}
\ieeecopyright{\copyright\ 2011 ITBA}
\lognumber{}
\pubitemident{}
\loginfo{15 de Marzo, 2012.}
\firstpage{1}

\confplacedate{Buenos Aires, Argentina, 15 de Marzo, 2012}

\maketitle               

\begin{abstract} 
El presente informe busca analizar y comparar distintas estrategias de búsqueda no informadas sobre un problema en particular haciendo uso de un motor de inferencias.
\end{abstract}

\begin{keywords}
DFS, BFS, General Problem Solver, depth-first, breadth-first, search strategy
\end{keywords}

%----------------------------------------------------------------------
% SECTION I: Introduccion%----------------------------------------------------------------------
\section{Introducción}

\PARstart Dentro de la optimización matemática podemos encontrar una técnica de análisis numérico denominada \textbf{cuadrados mínimos}. La misma intenta encontrar, dado un conjunto de pares (o ternas, etc.), la función que mejor se aproxime a los datos, de acuerdo con el criterio de mínimo error cuadrático.

En este caso se utilizará el método antes nombrado para encontrar una relación entre la velocidad promedio del viento y el tiempo en el instante de la medición. Estas velocidades corresponden a mediciones de doce estaciones meteorológicas situadas en la República de Irlanda, para el período 1971-1978.
En particular se busca que dicho ajuste venga dado por una función de la forma:

\begin{equation}
v(t)\ =\ A_0 + A_1\cos{(2\pi f_{1} t)} + B_1\sin{(2\pi f_{1} t)}
\end{equation}

donde $v$ es la velocidad del viento, $t$ es el tiempo en días y $f_1=\frac{1}{365.25}dia^{-1}$. Para corregir el desfase que existe con la duración del año trópico  \textit{(365 días 5 horas 48 minutos 45.25 segundos)}, se añade $0.25$ al la cantidad de días anuales, los cuales representan esas $\approx 6$ horas de diferencia.\\

Con este fin, se hará uso de tres métodos numéricos distintos (Cholesky, Gauss, QR) y luego se compararán los resultados obtenidos a fin de estudiar su complejidad, su error y consecuentemente su adecuación al problema planteado.

En primer lugar se expondrá un breve marco teórico, luego se explicará el proceso informático utilizado para manipular los datos, se presentarán los valores de los coeficientes $A_0$, $A_1$ y $B_1$ para cada estación meteorológica y se concluirá con un análisis previamente mencionados.


%----------------------------------------------------------------------
% SECTION II: Marco Teórico
%----------------------------------------------------------------------

\section{Marco Teórico}
\label{mat}
\subsection{Cuadrados Mínimos}

\par El problema de cuadrados mínimos consiste en encontrar el mínimo de la función

\begin{table}[H]
\begin{center}
\begin{tabular}{|c|c|c|}
\hline
Método & Tiempo medio & Tiempo  medio \\
 &  Octave $(sec)$ & Propio $(sec)$ \\
\hline
\hline
Eliminación Gaussiana & $0.011430$ & $0.013100$ \\
Cholesky & $0.009559$ & $0.012204$ \\
 QR &  $0.028912$&$0.031182$ \\
\hline  
\end{tabular}
\end{center}
\caption{Tabla de comparación de performance de los métodos propios versus los métodos nativos de GNU Octave}
\label{tmedio}
\end{table}

Estos resultados son consistentes con la siguiente tabla\footnote{Fierens, P. (2011), \emph{Cuadrados mínimos: repaso}, Buenos Aires: Instituto Tecnológico de Buenos Aires.}, la cual presenta el número de operaciones necesrias para resolver el problema de cuadrados mínimos cuando $A \in \Re ^{mxn}$:


\begin{table}[H]
\begin{center}
\begin{tabular}{|c|c|}
\hline
Algoritmo & Número de operaciones\\
\hline
\hline

Eliminación Gaussiana & $n^2(m+\frac{2n}{3})$\\
Cholesky & $n^2(m+\frac{n}{3})$\\
QR & $2n^2(m-\frac{n}{3})$\\

\hline  
\end{tabular}
\end{center}
\caption{Complejidad del problema de cuadrados mínimos}
\label{tmedio}
\end{table}

Si las columnas de la matriz A fueran casi linealmente dependientes, hubiera sido preferible la utilización del algoritmo \textbf{QR} frente a las ecuaciones normales. Como en este problema en particular esta condición no se cumple, el algoritmo más performante ha sido el de Cholesky.


%----------------------------------------------------------------------
% The bibliography. This bibliography was generated using the following
% two lines:
%\bibliographystyle{IEEEbib}
%\bibliography{ieeecls}
% where, the contents of the ieeecls.bib file was:
%
%@book{lamport,
%        AUTHOR = "Leslie Lamport",
%         TITLE = "A Document Preparation System: {\LaTeX} User's Guide
%                  and Reference Manual",
%       EDITION = "Second",
%     PUBLISHER = "Addison-Wesley",
%       ADDRESS = "Reading, MA",
%          YEAR = 1994,
%          NOTE = "Be sure to get the updated version for \LaTeX2e!"
%}
%
%@book{goossens,
%        AUTHOR = "Michel Goossens and Frank Mittelbach and
%                  Alexander Samarin",
%         TITLE = "The {\LaTeX} Companion",
%     PUBLISHER = "Addison-Wesley",
%       ADDRESS = "Reading, MA",
%          YEAR = 1994,
%}
%
% The ieeecls.bbl file was manually included here to make the distribution
% of this paper easier. You need not do it for your own papers.

\clearpage

\begin{thebibliography}{1}

\bibitem{lamport1}
Fierens, P. (2011),
\newblock {\em Cuadrados mínimos: repaso},
\newblock Buenos Aires: Instituto Tecnológico de Buenos Aires.

\bibitem{lamport1}
Abdi, H.,
\newblock {\em  Least-squares: {Encyclopedia for research methods for the social sciences}},
\newblock Thousand Oaks (CA): Sage. pp, 2003.

\bibitem{lamport1}
Farebrother, R.W. (1988),
\newblock {\em Linear Least Squares Computations, STATISTICS: Textbooks and Monographs}, \newblock New York: Marcel Dekker.

\bibitem{lamport1}
Lipson, M.; Lipschutz, S. (2001),
\newblock {\em Schaum's outline of theory and problems of linear algebra}, 
\newblock New York: McGraw-Hill, pp. 69–80.


\end{thebibliography}

%----------------------------------------------------------------------


\clearpage

\onecolumn

\onecolumn
\section*{Anexo A: Gráficos ilustrativos}

\begin{figure}[H]
\centering
%\includegraphics[scale=0.7]{./images/ciudadRPTconajuste.png}
\caption{Curva que ajusta las velocidades medidas en la estación RPT, obtenida mediante eliminación gaussiana.}
\label{dataRPT}
\end{figure}

\begin{figure}[H]
\centering
%\includegraphics[scale=0.7]{./images/CIUDAD.PNG}
\caption{Error de la curva de ajuste obtenida por eliminación gaussiana, con respecto a las velocidades de la estación RPT.}
\label{dataRPT}
\end{figure}

\clearpage



\section*{Anexo B: Código}

\subsection{Código para la obtención de las matrices $A$ y $b$}

%\VerbatimInput{./code/calculoAb.m}




\end{document}
