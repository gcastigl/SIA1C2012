
\documentclass[%
	%draft,
	%submission,
	%compressed,
	final,
	%
	%technote,
	%internal,
	%submitted,
	%inpress,
	reprint,
	%
	%titlepage,
	notitlepage,
	%anonymous,
	narroweqnarray,
	inline,
	twoside,
	invited
	]{ieee}

\usepackage[utf8]{inputenc}
\usepackage[spanish]{babel}
\usepackage{graphicx}
\usepackage{verbatim}
\usepackage{moreverb}
\usepackage{amsmath}
\usepackage{amsfonts}
\usepackage{amssymb}
\usepackage{fancybox}
\usepackage{float}
\usepackage{fancyvrb}
\usepackage{subfigure}

\newcommand{\latexiie}{\LaTeX2{\Large$_\varepsilon$}}

%\usepackage{ieeetsp}	% if you want the "trans. sig. pro." style
%\usepackage{ieeetc}	% if you want the "trans. comp." style
%\usepackage{ieeeimtc}	% if you want the IMTC conference style

% Use the `endfloat' package to move figures and tables to the end
% of the paper. Useful for `submission' mode.
%\usepackage {endfloat}

% Use the `times' package to use Helvetica and Times-Roman fonts
% instead of the standard Computer Modern fonts. Useful for the 
% IEEE Computer Society transactions.
%\usepackage{times}
% (Note: If you have the commercial package `mathtime,' (from 
% y&y (http://www.yandy.com), it is much better, but the `times' 
% package works too). So, if you have it...
%\usepackage {mathtime}

% for any plug-in code... insert it here. For example, the CDC style...
%\usepackage{ieeecdc}

\begin{document}

%----------------------------------------------------------------------
% Title Information, Abstract and Keywords
%----------------------------------------------------------------------
\title[Métodos de búsqueda no informados]{%
       Métodos de búsqueda no informados}

% format author this way for journal articles.
% MAKE SURE THERE ARE NO SPACES BEFORE A \member OR \authorinfo
% COMMAND (this also means `don't break the line before these
% commands).
\author[Castiglione, Karpovsky, Sturla]{Gonzalo V. Castiglione, Alan E. Karpovsky, Martín Sturla\\\textit{Estudiantes 
       Instituto Tecnológico de Buenos Aires (ITBA)}\\
\\\textbf{15 de Marzo de 2012}
}



\journal{Cátedra\ \ Sist.\ de\ Inteligencia\ Artificial,\ ITBA\ }
\titletext{-\ 15, MARZO\ 2012}
\ieeecopyright{\copyright\ 2011 ITBA}
\lognumber{}
\pubitemident{}
\loginfo{15 de Marzo, 2012.}
\firstpage{1}

\confplacedate{Buenos Aires, Argentina, 15 de Marzo, 2012}

\maketitle               

\begin{abstract} 
El presente informe busca analizar y comparar distintas estrategias de búsqueda no informadas sobre un problema en particular haciendo uso de un motor de inferencias.
\end{abstract}

\begin{keywords}
DFS, BFS, General Problem Solver, depth-first, breadth-first, search strategy
\end{keywords}

%----------------------------------------------------------------------
% SECTION I: Introduccion%----------------------------------------------------------------------
\section{Introducción}

\PARstart El juego \textit{Edificios} también conocido como \textit{Skyscraper puzzle} es una variante del conocido \textit{Sudoku} y consiste de una grilla cuadrada con números en su borde que representan las pistas sobre cuántos edificios se visualizan en esa dirección. El tablero, visto desde arriba, representa un espacio cubierto de edificios. Cada casillero debe ser completado con un dígito que va entre 1 y N, siendo N el tamaño de la grilla; haciendo que cada fila y cada columna contengan sólo una vez a cada dígito (como sucede en el Sudoku).

\par En este puzzle, cada dígito puesto en la grilla podría ser visualizado como un edificio de esa altura. Por ejemplo, si ingresamos un 5, estaríamos colocando un edificio de altura 5. Cada uno de los numeros que están por fuera de la grilla revelan la cantidad de edificios que pueden ser vistos al mirar la línea o columna en esa dirección. Cada edificio bloquea a todos los edificios de menor altura de ser vistos, mientras que los edificios de mayor altura son vistos a través de él.

%----------------------------------------------------------------------
% SECTION II: Marco Teórico
%----------------------------------------------------------------------

\section{Estados del problema}

\subsection{Estado inicial}

\par El estado inicial del problema es un tablero que contiene sólo las pistas en sus bordes (no contiene ningún edificio en la grilla). Para que el problema tenga solución éste tablero debe ser válido; es decir que no cualquier combinación de pistas sobre la visibilidad de edificios en esa dirección conducirán a un problema que sea posible resolver.

\subsection{Estado final}

\par El estado final del problema es un tablero con todos los casilleros completos (lleno de edificios) que cumpla con las raglas del juego citadas anteriormente.

\section{Modelado del problema}

\par Aaaaaa

\section{Reglas}

\par Aaaaaa

\section{Other}

\begin{table}[H]
\begin{center}
\begin{tabular}{|c|c|c|}
\hline
Método & Tiempo medio & Tiempo  medio \\
 &  Octave $(sec)$ & Propio $(sec)$ \\
\hline
\hline
Eliminación Gaussiana & $0.011430$ & $0.013100$ \\
Cholesky & $0.009559$ & $0.012204$ \\
 QR &  $0.028912$&$0.031182$ \\
\hline  
\end{tabular}
\end{center}
\caption{Tabla de comparación de performance de los métodos propios versus los métodos nativos de GNU Octave}
\label{tmedio}
\end{table}

Estos resultados son consistentes con la siguiente tabla\footnote{Fierens, P. (2011), \emph{Cuadrados mínimos: repaso}, Buenos Aires: Instituto Tecnológico de Buenos Aires.}, la cual presenta el número de operaciones necesrias para resolver el problema de cuadrados mínimos cuando $A \in \Re ^{mxn}$:


\begin{table}[H]
\begin{center}
\begin{tabular}{|c|c|}
\hline
Algoritmo & Número de operaciones\\
\hline
\hline

Eliminación Gaussiana & $n^2(m+\frac{2n}{3})$\\
Cholesky & $n^2(m+\frac{n}{3})$\\
QR & $2n^2(m-\frac{n}{3})$\\

\hline  
\end{tabular}
\end{center}
\caption{Complejidad del problema de cuadrados mínimos}
\label{tmedio}
\end{table}

Si las columnas de la matriz A fueran casi linealmente dependientes, hubiera sido preferible la utilización del algoritmo \textbf{QR} frente a las ecuaciones normales. Como en este problema en particular esta condición no se cumple, el algoritmo más performante ha sido el de Cholesky.


%----------------------------------------------------------------------
% The bibliography. This bibliography was generated using the following
% two lines:
%\bibliographystyle{IEEEbib}
%\bibliography{ieeecls}
% where, the contents of the ieeecls.bib file was:
%
%@book{lamport,
%        AUTHOR = "Leslie Lamport",
%         TITLE = "A Document Preparation System: {\LaTeX} User's Guide
%                  and Reference Manual",
%       EDITION = "Second",
%     PUBLISHER = "Addison-Wesley",
%       ADDRESS = "Reading, MA",
%          YEAR = 1994,
%          NOTE = "Be sure to get the updated version for \LaTeX2e!"
%}
%
%@book{goossens,
%        AUTHOR = "Michel Goossens and Frank Mittelbach and
%                  Alexander Samarin",
%         TITLE = "The {\LaTeX} Companion",
%     PUBLISHER = "Addison-Wesley",
%       ADDRESS = "Reading, MA",
%          YEAR = 1994,
%}
%
% The ieeecls.bbl file was manually included here to make the distribution
% of this paper easier. You need not do it for your own papers.

\clearpage

\begin{thebibliography}{1}

\bibitem{lamport1}
Fierens, P. (2011),
\newblock {\em Cuadrados mínimos: repaso},
\newblock Buenos Aires: Instituto Tecnológico de Buenos Aires.

\bibitem{lamport1}
Abdi, H.,
\newblock {\em  Least-squares: {Encyclopedia for research methods for the social sciences}},
\newblock Thousand Oaks (CA): Sage. pp, 2003.

\bibitem{lamport1}
Farebrother, R.W. (1988),
\newblock {\em Linear Least Squares Computations, STATISTICS: Textbooks and Monographs}, \newblock New York: Marcel Dekker.

\bibitem{lamport1}
Lipson, M.; Lipschutz, S. (2001),
\newblock {\em Schaum's outline of theory and problems of linear algebra}, 
\newblock New York: McGraw-Hill, pp. 69–80.


\end{thebibliography}

%----------------------------------------------------------------------


\clearpage

\onecolumn

\onecolumn
\section*{Anexo A: Gráficos ilustrativos}

\begin{figure}[H]
\centering
%\includegraphics[scale=0.7]{./images/ciudadRPTconajuste.png}
\caption{Curva que ajusta las velocidades medidas en la estación RPT, obtenida mediante eliminación gaussiana.}
\label{dataRPT}
\end{figure}

\begin{figure}[H]
\centering
%\includegraphics[scale=0.7]{./images/CIUDAD.PNG}
\caption{Error de la curva de ajuste obtenida por eliminación gaussiana, con respecto a las velocidades de la estación RPT.}
\label{dataRPT}
\end{figure}

\clearpage



\section*{Anexo B: Código}

\subsection{Código para la obtención de las matrices $A$ y $b$}

%\VerbatimInput{./code/calculoAb.m}




\end{document}
